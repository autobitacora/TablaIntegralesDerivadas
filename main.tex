\documentclass[a4paper,10pt,openany]{book}
\usepackage[spanish]{babel}
\usepackage[utf8]{inputenc}
%\usepackage{graphicx}
%\usepackage[backend=biber]{biblatex}
\usepackage{fullpage}
\usepackage{amsmath,amssymb,enumitem}
%\usepackage[dvipsnames]{xcolor}
%\usepackage{tikz}
%\usetikzlibrary{shapes,arrows}

% Eliminamos los sangrados
\setlength{\parindent}{0cm}
% Asignamos el archivo referencias a la bibliografia
%\bibliography{referencias}
% Declaramos las funciones matematicas que faltan en amsmath
\DeclareMathOperator{\arcsec}{arcsec}
\DeclareMathOperator{\arccot}{arccot}
\DeclareMathOperator{\arccsc}{arccsc}
\DeclareMathOperator{\csch}{csch}
\DeclareMathOperator{\argsenh}{argsenh}
\DeclareMathOperator{\argcosh}{argcosh}
\DeclareMathOperator{\argtanh}{argtanh}
\DeclareMathOperator{\argcoth}{argcoth}
\DeclareMathOperator{\argsech}{argsech}
\DeclareMathOperator{\argcsch}{argcsch}

\begin{document}

\author{Raul Marusca}
\title{Tabla de Integrales y Derivadas}
\date{}

%\frontmatter
\maketitle
\tableofcontents

%\mainmatter
\chapter{Derivadas}
\section{Definici\'on de derivada}
La derivada de una funci\'on de $x$ respecto de la variable $x$, si existe,  es igual al limite del cociente incremental. El cociente incremental se determina evaluando el valor de la funcion en un punto mas un diferencial y restando el valor que toma la funci\'on en ese punto.
%\begin{figure}[!h]
%	\centering
%\begin{tikzpicture}[line cap=round,line join=round,>=stealth',x=1.0cm,y=1.0cm,domain=-0.5:2.25]
%	\draw[->,color=black] (-1,0) -- (4,0);
%	\foreach \x in {1,2,3}
%	\draw[shift={(\x,0)},color=black] (0pt,2pt) -- (0pt,-2pt) node[below] {\footnotesize $\x$};
%	\draw[color=black] (4,0) node [right] { $x$};
%	\draw[->,color=black] (0,-1) -- (0,5);
%	\foreach \y in {1,2,3,4}
%	\draw[shift={(0,\y)},color=black] (2pt,0pt) -- (-2pt,0pt) node[left] {\footnotesize $\y$};
%	\draw[color=black] (0,5) node [above] { $y$};
%	\draw[color=red,<->] plot[id=quad] function{x*x} node[right] {$f(x) =x^2$};
%	\fill[black] (0.5,0.25) circle (1.25pt);
%	\fill[black] (1.5,2.25) circle (1.25pt);
%	\draw (0.5,0.25)--(1.5,2.25)--(1.5,0.25)--cycle;
%	\draw[|-|,yshift=-0.25cm] (0.5,0.25)--(1.5,0.25);
%	\draw[yshift=-0.25cm](1,0.25) node[fill=white] {dx};
%	\draw[|-|,xshift=0.25cm] (1.5,0.25)--(1.5,2.25);
%	\draw[xshift=0.25cm] (1.5,1.25) node [fill=white] {dy};
%\end{tikzpicture}
%\end{figure}
%La pendiente de la recta tangente a la funci\'on en el punto es:
%\begin{equation*}
%	m=\frac{dy}{dx}
%\end{equation*}
\begin{equation*}
	\frac{df(x)}{dx}=\lim_{\Delta x \rightarrow 0}\frac{\Delta f(x)}{\Delta x}=\lim_{\Delta x \rightarrow 0}\frac{f(x_0 +\Delta x)-f(x_o)}{\Delta x}
\end{equation*}
\section{Generalidades}
Derivada de una constante
\begin{equation*}
	\frac{d}{dx}(c)=0
\end{equation*}
Derivada de la constante por la variable
\begin{equation*}
	\frac{d}{dx}(cx)=c
\end{equation*}
Derivada de la variable a una potencia
\begin{equation*}
	\frac{d}{dx}(x^n)=nx^{n-1}
\end{equation*}
Derivada de una constante por la variable a una potencia
\begin{equation*}
	\frac{d}{dx}(cx^n)=ncx^{n-1}
\end{equation*}
Derivada de la suma de funciones de x\\
($u$, $v$, $w$ son funciones de x)
\begin{equation*}
	\frac{d}{dx}(u\pm v\pm w \pm ...)=\frac{du}{dx}\pm\frac{dv}{dx}\pm\frac{dw}{dx}\pm...
\end{equation*}
Derivada de la constante por una funcion de x
\begin{equation*}
	\frac{d}{dx}(cu)=c\frac{du}{dx}
\end{equation*}
\chapter{Integrales Indefinidas}

\section{Propiedades y Generalidades}

\section{Algunas soluciones importantes}

\section{Integrales que contienen $ax + b$}

\section{Integrales que contienen $\sqrt{ax + b}$}

\section{Integrales que contienen $ax + b$ y $px + q$}

\section{Integrales que contienen $\sqrt{ax + b}$ y $px + q$}

\section{Integrales que contienen $\sqrt{ax + b}$ y $\sqrt{px + q}$}

\section{Integrales que contienen $x^2 + a^2$}

\section{Integrales que contienen $x^2 - a^2$, para $x^2 \ge a^2$}

\section{Integrales que contienen $a^2 + x^2$}

\section{Integrales que contienen $\sqrt{x^2 + a^2}$}

\section{Integrales que contienen $\sqrt{x^2 - a^2}$}

\section{Integrales que contienen $\sqrt{a^2 - x^2}$}

\section{Integrales que contienen $ax^2 + bx + c$}

\section{Integrales que contienen $\sqrt{ax^2 + bx + c}$}

\section{Integrales que contienen $x^3 + a^3$}

\section{Integrales que contienen $x^4 + a^4$}

\section{Integrales que contienen $x^4 - a^4$}

\section{Integrales que contienen $x^n \pm a^n$}

\section{Integrales que contienen $\sin ax$}

\section{Integrales que contienen $\cos ax$}

\section{Integrales que contienen $\sin ax$ y $\cos ax$}

\section{Integrales que contienen $\tan ax$}

\section{Integrales que contienen $\cot ax$}

\section{Integrales que contienen $\sec ax$}

\section{Integrales que contienen $\csc ax$}

\section{Integrales que contienen funciones trigonométricas inversas}
\subsection{$\arcsin $}
\subsection{$\arccos $}
\subsection{$\arctan $}
\subsection{$\arccot $}
\subsection{$\arccsc $}

\section{Integrales que contienen $\ln ax$}

\section{Integrales que contienen $ e^{nx}$}

\section{Integrales que contienen $\sinh ax$}

\section{Integrales que contienen $\cosh ax$}

\section{Integrales que contienen $\sinh ax$ y $\cosh ax$}

\section{Integrales que contienen $\tanh ax$}

\section{Integrales que contienen $\coth ax$}

\section{Integrales que contienen $\csch ax$}

\section{Integrales que contienen funciones hiperbólicas inversas}
\subsection{$\argsenh $}
\subsection{$\argcosh $}
\subsection{$\argtanh $}
\subsection{$\argcoth $}
\subsection{$\argsech $}
\subsection{$\argcsch $}
\chapter{Integrales Definidas}

\section{Definición}

\section{Soluciones para integrales impropias}

\section{Algunas propiedades}

\section{Integrales definidas que contienen funciones trigonométricas}

\section{Integrales definidas que contienen funciones exponenciales}

\section{Integrales definidas que contienen funciones logarítmicas}

\section{Integrales definidas que contienen funciones hiperbólicas}

\section{Integrales definidas que contienen funciones racionales e irracionales}


\chapter{Apéndice}

\section{Funciones trigonométricas}
\subsection{Interpretación grafica}
\subsection{Relaciones útiles}
\section{Funciones hiperbolicas}
\subsection{Interpretación grafica}
\subsection{Relaciones útiles}

\section{Constantes notables}

%\backmatter
%\printbibliography

\end{document}