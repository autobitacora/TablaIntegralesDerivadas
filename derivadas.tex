\chapter{Derivadas}
\section{Definición de derivada}
La derivada de una función de $x$ respecto de la variable $x$, si existe,  es igual al limite del cociente incremental. El cociente incremental se determina evaluando el valor de la función en un punto mas un diferencial y restando el valor que toma la función en ese punto.
%\begin{figure}[!h]
%	\centering
%\begin{tikzpicture}[line cap=round,line join=round,>=stealth',x=1.0cm,y=1.0cm,domain=-0.5:2.25]
%	\draw[->,color=black] (-1,0) -- (4,0);
%	\foreach \x in {1,2,3}
%	\draw[shift={(\x,0)},color=black] (0pt,2pt) -- (0pt,-2pt) node[below] {\footnotesize $\x$};
%	\draw[color=black] (4,0) node [right] { $x$};
%	\draw[->,color=black] (0,-1) -- (0,5);
%	\foreach \y in {1,2,3,4}
%	\draw[shift={(0,\y)},color=black] (2pt,0pt) -- (-2pt,0pt) node[left] {\footnotesize $\y$};
%	\draw[color=black] (0,5) node [above] { $y$};
%	\draw[color=red,<->] plot[id=quad] function{x*x} node[right] {$f(x) =x^2$};
%	\fill[black] (0.5,0.25) circle (1.25pt);
%	\fill[black] (1.5,2.25) circle (1.25pt);
%	\draw (0.5,0.25)--(1.5,2.25)--(1.5,0.25)--cycle;
%	\draw[|-|,yshift=-0.25cm] (0.5,0.25)--(1.5,0.25);
%	\draw[yshift=-0.25cm](1,0.25) node[fill=white] {dx};
%	\draw[|-|,xshift=0.25cm] (1.5,0.25)--(1.5,2.25);
%	\draw[xshift=0.25cm] (1.5,1.25) node [fill=white] {dy};
%\end{tikzpicture}
%\end{figure}
%La pendiente de la recta tangente a la funci\'on en el punto es:
%\begin{equation*}
%	m=\frac{dy}{dx}
%\end{equation*}
\begin{equation*}
	\frac{df(x)}{dx}=\lim_{\Delta x \rightarrow 0}\frac{\Delta f(x)}{\Delta x}=\lim_{\Delta x \rightarrow 0}\frac{f(x_0 +\Delta x)-f(x_o)}{\Delta x}
\end{equation*}
\section{Generalidades}
Derivada de una constante
\begin{equation}
	\frac{d}{dx}(c)=0
\end{equation}
Derivada de la constante por la variable
\begin{equation}
	\frac{d}{dx}(cx)=c
\end{equation}
Derivada de la variable a una potencia
\begin{equation}
	\frac{d}{dx}(x^n)=nx^{n-1}
\end{equation}
Derivada de una constante por la variable a una potencia
\begin{equation}
	\frac{d}{dx}(cx^n)=ncx^{n-1}
\end{equation}
Derivada de la suma de funciones de x\\
($u$, $v$, $w$ son funciones de x)
\begin{equation}
	\frac{d}{dx}(u\pm v\pm w \pm ...)=\frac{du}{dx}\pm\frac{dv}{dx}\pm\frac{dw}{dx}\pm...
\end{equation}
Derivada de la constante por una función de x
\begin{equation}
	\frac{d}{dx}(cu)=c\frac{du}{dx}
\end{equation}