\chapter{Derivadas}
\section{Definición de derivada}
La derivada de una función de $x$ respecto de la variable $x$, si existe,  es igual al limite del cociente incremental. El cociente incremental se determina evaluando el valor de la función en un punto mas un diferencial y restando el valor que toma la función en ese punto.
%\begin{figure}[!h]
%	\centering
%\begin{tikzpicture}[line cap=round,line join=round,>=stealth',x=1.0cm,y=1.0cm,domain=-0.5:2.25]
%	\draw[->,color=black] (-1,0) -- (4,0);
%	\foreach \x in {1,2,3}
%	\draw[shift={(\x,0)},color=black] (0pt,2pt) -- (0pt,-2pt) node[below] {\footnotesize $\x$};
%	\draw[color=black] (4,0) node [right] { $x$};
%	\draw[->,color=black] (0,-1) -- (0,5);
%	\foreach \y in {1,2,3,4}
%	\draw[shift={(0,\y)},color=black] (2pt,0pt) -- (-2pt,0pt) node[left] {\footnotesize $\y$};
%	\draw[color=black] (0,5) node [above] { $y$};
%	\draw[color=red,<->] plot[id=quad] function{x*x} node[right] {$f(x) =x^2$};
%	\fill[black] (0.5,0.25) circle (1.25pt);
%	\fill[black] (1.5,2.25) circle (1.25pt);
%	\draw (0.5,0.25)--(1.5,2.25)--(1.5,0.25)--cycle;
%	\draw[|-|,yshift=-0.25cm] (0.5,0.25)--(1.5,0.25);
%	\draw[yshift=-0.25cm](1,0.25) node[fill=white] {dx};
%	\draw[|-|,xshift=0.25cm] (1.5,0.25)--(1.5,2.25);
%	\draw[xshift=0.25cm] (1.5,1.25) node [fill=white] {dy};
%\end{tikzpicture}
%\end{figure}
%La pendiente de la recta tangente a la funci\'on en el punto es:
%\begin{equation*}
%	m=\frac{dy}{dx}
%\end{equation*}
\begin{equation*}
	\frac{df(x)}{dx}=\lim_{\Delta x \rightarrow 0}\frac{\Delta f(x)}{\Delta x}=\lim_{\Delta x \rightarrow 0}\frac{f(x_0 +\Delta x)-f(x_o)}{\Delta x}
\end{equation*}
\section{Generalidades}
Derivada de una constante
\begin{equation}
	\frac{d}{dx}(c)=0
\end{equation}
Derivada de la constante por la variable
\begin{equation}
	\frac{d}{dx}(cx)=c
\end{equation}
Derivada de la variable a una potencia
\begin{equation}
	\frac{d}{dx}(x^n)=nx^{n-1}
\end{equation}
Derivada de una constante por la variable a una potencia
\begin{equation}
	\frac{d}{dx}(cx^n)=ncx^{n-1}
\end{equation}
Derivada de la suma de funciones de x\\
($u$, $v$, $w$ son funciones de x)
\begin{equation}
	\frac{d}{dx}(u\pm v\pm w \pm ...)=\frac{du}{dx}\pm\frac{dv}{dx}\pm\frac{dw}{dx}\pm...
\end{equation}
Derivada de la constante por una función de x
\begin{equation}
	\frac{d}{dx}(cu)=c\frac{du}{dx}
\end{equation}
Derivada del producto de dos funciones de x
\begin{equation}
	\frac{d}{dx}(uv)=u\frac{dv}{dx}+v\frac{du}{dx}
\end{equation}
Derivada del producto de múltiples funciones de x
\begin{equation}
	\frac{d}{dx}(uvw)=uv\frac{dw}{dx}+uw\frac{dv}{dx}+vw\frac{du}{dx}
\end{equation}
Derivada del cociente de dos funciones de x
\begin{equation}
	\frac{d}{dx}\frac{u}{v}=\frac{v\frac{du}{dx}-u\frac{dv}{dx}}{v^2}
\end{equation}
Regla de la cadena
\begin{equation}
	\frac{dy}{dx}=\frac{dy}{du}\frac{du}{dx}
\end{equation}
Derivada de función elevada a una potencia (Regla de la cadena)
\begin{equation}
	\frac{d}{dx}(u^n)=nu^{n-1}\frac{du}{dx}
\end{equation}
Relación inversa de diferenciales
\begin{equation}
	\frac{du}{dx}=\frac{1}{\frac{dx}{du}}
\end{equation}
Regla para simplificar diferenciales (variable intermedia)
\begin{equation}
	\frac{dy}{dx}=\frac{\frac{dy}{du}}{\frac{dx}{du}}
\end{equation}
\section{Derivadas de funciones trigonométricas}
Derivada del seno
\begin{equation}
	\frac{d}{dx}\sin u = \cos u \frac{du}{dx}
\end{equation}
Derivada del coseno
\begin{equation}
	\frac{d}{dx}\cos u = -\sin u \frac{du}{dx}
\end{equation}
Derivada de la tangente
\begin{equation}
	\frac{d}{dx}\tan u = \sec^2 u \frac{du}{dx}
\end{equation}
Derivada de la cotangente
\begin{equation}
	\frac{d}{dx}\cot u = -\csc^2 u \frac{du}{dx}
\end{equation}
Derivada de la secante
\begin{equation}
	\frac{d}{dx}\sec u = \sec u \tan u \frac{du}{dx}
\end{equation}
Derivada de la cosecante
\begin{equation}
	\frac{d}{dx}\csc u = -\csc u \cot u \frac{du}{dx}
\end{equation}
\section{Derivadas de funciones trigonométricas inversas}
Derivada del arcoseno
\begin{equation}
	\frac{d}{du} \arcsin u = \frac{1}{\sqrt{1-u^2}}\frac{du}{dx}\qquad -\frac{\pi}{2} < \arcsin\; u < \frac{\pi}{2}
\end{equation}
Derivada del arcocoseno
\begin{equation}
	\frac{d}{du} \arccos u = -\frac{1}{\sqrt{1-u^2}}\frac{du}{dx}\qquad 0 < \arccos\; u < \pi
\end{equation}
Derivada del arcotangente
\begin{equation}
	\frac{d}{du} \arctan u = \frac{1}{1+u^2}\frac{du}{dx}\qquad -\frac{\pi}{2} < \arctan\; u < \frac{\pi}{2}
\end{equation}
Derivada del arcocotangente
\begin{equation}
	\frac{d}{du} \arccot u = -\frac{1}{1+u^2}\frac{du}{dx}\qquad 0 < \arccot\; u < \pi
\end{equation}
Derivada del arcosecante
\begin{equation}
	\frac{d}{du} \arcsec u = \pm \frac{1}{u\sqrt{u^2-1}}\frac{du}{dx}\qquad\left\{
	\begin{array}{ll}
		+\; Si  & 0\; < \arcsec u < \frac{\pi}{2}  \\
		-\; Si  & \frac{\pi}{2} < \arcsec u < \pi
	\end{array}
	\right.
\end{equation}
Derivada del arcocosecante
\begin{equation}
	\frac{d}{du} \arccsc u = \mp \frac{1}{u\sqrt{u^2-1}}\frac{du}{dx}\qquad\left\{
	\begin{array}{ll}
		-\; Si  & \quad 0 < \arccsc u < \frac{\pi}{2}  \\
		+\; Si  & -\frac{\pi}{2} < \arccsc u < 0
	\end{array}
	\right.
\end{equation}
\section{Derivadas de funciones logarítmicas y exponenciales}
Derivada del logaritmo en base $a$ de una función de $x$
\begin{equation}
	\frac{d}{dx}\log_a u=\frac{\log_a}{u}\frac{du}{dx} 
\end{equation}
Derivada del logaritmo natural de una función de $x$
\begin{equation}
	\frac{d}{dx}\ln u=\frac{1}{u}\frac{du}{dx} 
\end{equation}
Derivada la base elevada a una función de $x$
\begin{equation}
	\frac{d}{dx}a^u=a^u \ln a \frac{du}{dx} 
\end{equation}
Derivada la base $e$ elevada a una función de $x$
\begin{equation}
	\frac{d}{dx}e^u=e^u \frac{du}{dx} 
\end{equation}
Derivada de una función de $x$ elevada a otra función de $x$
\begin{equation}
	\frac{d}{dx}u^v=\frac{d}{dx}e^{v \ln u}=e^{v \ln u} \frac{d}{dx}(v \ln u)=vu^{v^{-1}}\frac{du}{dx}+u^v \ln u \frac{dv}{dx} 
\end{equation}
\section{Derivadas de funciones hiperbólicas}

\section{Derivadas de funciones hiperbólicas inversas}